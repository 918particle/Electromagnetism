\title{Warm-Up for April 20th, 2022}
\author{Dr. Jordan Hanson - Whittier College Dept. of Physics and Astronomy}
\date{\today}
\documentclass[12pt]{article}
\usepackage[a4paper, total={18cm, 27cm}]{geometry}
\usepackage{graphicx}
\usepackage{amsmath}
\usepackage{subcaption}
\usepackage{bm}
\def\rcurs{{\mbox{$\resizebox{.16in}{.08in}{\includegraphics{ScriptR}}$}}}
\def\brcurs{{\mbox{$\resizebox{.16in}{.08in}{\includegraphics{BoldR}}$}}}
\def\hrcurs{{\mbox{$\hat \brcurs$}}}
 
\begin{document}
\maketitle
\small
\section{Memory Bank}
\begin{enumerate}
\item Magnetic dipole field, assuming $\mathbf{m} = m\hat{\mathbf{z}}$ ($\mathbf{m} = I\int d\mathbf{a}$):
\begin{equation}
\mathbf{B} = \frac{\mu_0 m}{4\pi r^3}\left(2\cos\theta \hat{\mathbf{r}} + \sin\theta \hat{\boldsymbol{\theta}}\right)
\end{equation}
\item Torque on a magnetic dipole:
\begin{equation}
\boldsymbol{\tau} = \boldsymbol{\mu} \times \mathbf{B}
\end{equation}
\end{enumerate}

\section{Magnetic Dipole Moment, and Dipole Field}

\begin{enumerate}
\item A circular loop of wire, with radius $R$, lies in the $xy$ plane (centered at the origin) and carries a current $I$ running counterclockwise as viewed from the positive $z$ axis. (a) What is the magnetic dipole moment? (b) What is the approximate magnetic field at points far from the origin? (c) Show that, for points on the $z$ axis, your answer is consistent with the \textit{exact} field when $z\gg R$. (d) Compute the torque on the system in the presence of an external $\mathbf{B}$-field: $\mathbf{B} = B_0 \hat{\mathbf{y}}$.
\end{enumerate}


\end{document}