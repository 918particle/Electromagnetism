\title{Warm-Up for February 25th, 2022}
\author{Dr. Jordan Hanson - Whittier College Dept. of Physics and Astronomy}
\date{\today}
\documentclass[12pt]{article}
\usepackage[a4paper, total={18cm, 27cm}]{geometry}
\usepackage{graphicx}
\usepackage{amsmath}
\usepackage{bm}
\begin{document}
\maketitle

\section{Memory Bank}

\begin{enumerate}
\item Curl of $\mathbf{E}$-fields: $\nabla \times \mathbf{E} = 0$.
\item Potential of the $\mathbf{E}$-field: $V(\mathbf{r}) = - \int_{\mathcal{O}}^{\mathbf{r}} \mathbf{E} \cdot d\mathbf{l}$.  We usually take $\mathcal{O}$ to be infinitely far from charge.
\end{enumerate}

\section{Multiple Choice}

Which of the following are $\mathbf{E}$-fields?

\begin{itemize}
\item A: $\mathbf{E}_1 = a y \hat{x} + b x \hat{y}$
\item B: $\mathbf{E}_2 = a y \hat{x} + a x \hat{y}$
\item C: $\mathbf{E}_3 = a \hat{x} + b \hat{y}$
\item D: $\mathbf{E}_4 = \frac{k q}{y^2}\hat{y}$
\end{itemize}

\section{Exercise}

Find the potential (a) outside and (b) inside a spherical shell of radius $R$ that carries a uniform surface charge.  First use Gauss' law to obtain the field in each region, then use the definition of $V(\mathbf{r})$ properly.

\end{document}