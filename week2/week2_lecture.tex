\documentclass{beamer}
\usetheme{metropolis}
\usepackage{graphicx}
\usepackage{amsmath}
\usepackage{tcolorbox}
\title{Electromagnetc Theory: PHYS330}
\author{Jordan Hanson}
\institute{Whittier College Department of Physics and Astronomy}

\begin{document}
\maketitle

\section{Summary}

\begin{frame}{Week 2 Summary}
\begin{enumerate}
\item Homework discussions
\begin{itemize}
\item Proofs!  Glorious proofs.
\item Exercises with \textit{checking} fundamental theorems
\end{itemize}
\item Electrostatics and Coulomb forces
\begin{itemize}
\item Charge distributions, superposition, and the Coulomb force
\item A note about the \textit{far-field}
\item Setting up integrals, taking limits, checking units
\item The divergence of electric fields
\item The curl of electric fields
\end{itemize}
\item Electric Potential
\begin{itemize}
\item Definitions, fundamental theorem for gradients
\item Reference points
\item Laplace equation ...
\end{itemize}
\item Work, energy, and conductors
\end{enumerate}
\end{frame}

\section{Homework}

\begin{frame}{Homework, Week 2}
Unlike last week, these exercises come from \textit{within} the chapter.  Ideally, you should look at all of the problems within the chapter as you study.
\begin{itemize}
\item Exercise 2.5
\item Exercise 2.6
\item Exercise 2.9
\item Exercise 2.12
\item Exercsie 2.16
\item Exercise 2.18
\item Exercise 2.25
\item Exercise 2.29
\end{itemize}
\end{frame}

\section{Charge distributions, Superposition, and the Coulomb Force}

\begin{frame}{Charge distributions, Superposition, and the Coulomb Force}
\begin{figure}
\centering
\includegraphics[width=5cm]{figures/2_1.jpg}
\includegraphics[width=5cm]{figures/2_3.jpg}
\caption{\label{fig:2_1} The basic problem of electrostatics. Note the definition of the separation vector, and the vectors to the field point and to all the source charges.}
\end{figure}
\end{frame}

\begin{frame}{Charge distributions, Superposition, and the Coulomb Force}
\begin{figure}
\centering
\includegraphics[width=10cm]{figures/2_4.jpg}
\caption{\label{fig:2_4} Begin with a dipole, and then a \textit{physical} dipole.}
\end{figure}
\end{frame}

\begin{frame}{Charge distributions, Superposition, and the Coulomb Force}
\begin{figure}
\centering
\includegraphics[width=8cm]{figures/2_5.jpg}
\caption{\label{fig:2_5} The continuous limit implies a variety of symmetries and geometries over which we integrate, rather than sum.}
\end{figure}
\end{frame}

\begin{frame}{Charge distributions, Superposition, and the Coulomb Force}
\begin{figure}
\centering
\includegraphics[width=6cm]{figures/2_6.jpg}
\caption{\label{fig:2_6} A coninuous line density of charge.  Integration yields the electric field.}
\end{figure}
\end{frame}

\end{document}
