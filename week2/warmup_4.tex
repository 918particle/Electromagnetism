\title{Warm-Up for February 16th, 2022}
\author{Dr. Jordan Hanson - Whittier College Dept. of Physics and Astronomy}
\date{\today}
\documentclass[10pt]{article}
\usepackage[a4paper, total={18cm, 27cm}]{geometry}
\usepackage{outlines}
\usepackage{hyperref}
\usepackage{amsmath}
\usepackage{bm}
\begin{document}
\maketitle

\section{Electrostatics}

\begin{enumerate}
\item (a) Find the electric field \textbf{E} due to the presence of two charges of strength $q$, located distances $\pm d/2$ from the origin, at a distance $z$ above the origin. (b) What is the \textbf{E}-field at a position \textbf{x}, where $|$\textbf{x}$| > d/2$? \\ \vspace{5cm}
\item Suppose there is an external field \textbf{E}$_0 = E_0 \hat{z}$.  What is the torque on the source charges? \\ \vspace{2cm}
\item (a) Suppose the right charge is negative, and the left charge is positive.  What is the new \textbf{E}-field at \textit{P}? (b) Let \textbf{p} $= q$\textbf{d}, where \textbf{d} points from the positive to the negative charge.  Recall that the torque on a system at \textbf{r}, rotated about the origin by a force \textbf{F} is $\bm{\tau}$ = \textbf{r}$ \times $\textbf{F}.  Show that the torque on the system is $\bm{\tau}$ = \textbf{p}$ \times $\textbf{E}$_0$.
\end{enumerate}

\end{document}