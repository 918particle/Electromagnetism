\title{Warm-Up for March 4th, 2022}
\author{Dr. Jordan Hanson - Whittier College Dept. of Physics and Astronomy}
\date{\today}
\documentclass[12pt]{article}
\usepackage[a4paper, total={18cm, 27cm}]{geometry}
\usepackage{graphicx}
\usepackage{amsmath}
\usepackage{bm}
\begin{document}
\maketitle

\section{Memory Bank}

\begin{enumerate}
\item The \textbf{Fourier series} representation of a function $f(x)$ is written:
\begin{equation}
S(x) = \frac{A_0}{2}+\sum_{i=1}^{\infty} \left( A_n \cos(nx) + B_n \sin(nx) \right)
\end{equation}
with
\begin{align}
A_n &= \frac{1}{\pi} \int_0^{2\pi} f(x) \cos(nx) dx \\
B_n &= \frac{1}{\pi} \int_0^{2\pi} f(x) \sin(nx) dx
\end{align}
\end{enumerate}

\section{Representing a Solution with a Fourier Series}

Suppose we have an arrangement of charge such that we can create a \textit{periodic} potential:

\begin{equation}
f(x) = x, ~~ 0 \leq x \leq 2\pi
\end{equation}

Between $2\pi$ and $4\pi$, $f(x) = (x-2\pi)$, etc., so that the function repeats.  (a) Use the formulas in the memory bank to determine $A_n$ and $B_n$.  (b) Can you graph the series and the function together, to see if they match?

\end{document}