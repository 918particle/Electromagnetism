\title{Warm-Up for $\pi$-Day, 2022}
\author{Dr. Jordan Hanson - Whittier College Dept. of Physics and Astronomy}
\date{\today}
\documentclass[12pt]{article}
\usepackage[a4paper, total={18cm, 27cm}]{geometry}
\usepackage{graphicx}
\usepackage{amsmath}
\usepackage{bm}
\def\rcurs{{\mbox{$\resizebox{.16in}{.08in}{\includegraphics{ScriptR}}$}}}
\def\brcurs{{\mbox{$\resizebox{.16in}{.08in}{\includegraphics{BoldR}}$}}}
\def\hrcurs{{\mbox{$\hat \brcurs$}}}
 
\begin{document}
\maketitle

\section{Memory Bank}

\begin{enumerate}
\item Recall the definition of the Taylor series, in which $f^{(n)}(a)$ is the $n$-th derivative of a function $f(x)$ evaluated at $x = a$:
\begin{equation}
f(x) = \sum_{n=0}^{\infty} \frac{f^{(n)}(a)}{n!}(x-a)^n
\end{equation}
\item The Law of Cosines, with sides of lengths $a$, $b$, and $c$, and with angle $\gamma$ between sides $a$ and $b$, states that
\begin{equation}
c^2 = a^2 + b^2 - 2 a b \cos\gamma
\end{equation}
\end{enumerate}

\section{Tools for the Multipole Expansion}

\begin{enumerate}
\item Find the Taylor series up to $\mathcal{O}(x^2)$ for $f(x) = 1/\sqrt{1+x}$ near $x = 0$. \\ \vspace{3cm}
\item Recall that the definition of displacement between charge and observer is
\begin{equation}
\brcurs = \mathbf{r} - \mathbf{r}'
\end{equation}
Find the magnitude squared of $\brcurs$ to reveal the Law of Cosines.
\end{enumerate}

\end{document}