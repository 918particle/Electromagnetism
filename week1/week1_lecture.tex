\documentclass{beamer}
\usetheme{metropolis}
\usepackage{graphicx}
\usepackage{amsmath}
\usepackage{tcolorbox}
\title{Electromagnetc Theory: PHYS330}
\author{Jordan Hanson}
\institute{Whittier College Department of Physics and Astronomy}

\begin{document}
\maketitle

\section{Summary}

\begin{frame}{Summary}
\begin{enumerate}
\item Electromagnetism and the module system
\begin{itemize}
\item Pace
\item Style
\item Class decision
\end{itemize}
\item Challenge level: pre-requisites
\begin{itemize}
\item Passed Calculus 1, 2, and 3
\item Passed Calculus-based physics 1, 2, and 3
\item Passed modern physics
\end{itemize}
\item Maxwell's equations live in 3D
\item \alert{Introduction to Electromagnetism by D. Griffiths (4th ed.)}
\item First half of the text is recommended by publisher, retain for graduate school
\item Asynchronous content: www.youtube.com/918particle, and Moodle in folders
\end{enumerate}
\end{frame}

\section{Homework}

\begin{frame}{Homework}
\begin{enumerate}
\item Reading: Chapter 1 by Friday/Saturday
\item Exercises: 1.54, 1.55, 1.56, 1.57, 1.59, 1.62, 1.63, 1.64
\end{enumerate}
\end{frame}

\section{Today: the Dirac delta-function}

\begin{frame}{The Dirac $\delta$-function}
Consider this function:
\begin{equation}
\vec{v} = \frac{1}{r^2}\hat{r}
\end{equation}
with $\vec{r} = x\hat{i} + y\hat{j} + z\hat{k}$.  What is the divergence?
\begin{equation}
\nabla \cdot \vec{v} = \frac{1}{r^2}\frac{\partial}{\partial r}(r^2 v_r) + \frac{1}{r\sin(\theta)}\frac{\partial}{\partial \theta}(r\sin(\theta) v_\theta) + \frac{1}{r\sin(\theta)} \frac{\partial v_\phi}{\partial \phi}
\end{equation}
\end{frame}

\begin{frame}{The Dirac $\delta$-function}
So we find the divergence is zero.  What is the result of a surface integral around the origin?
\begin{equation}
\oint \vec{v} \cdot d \vec{a} = \int_0^{2\pi} \int_0^{\pi} \left(\frac{\hat{r}}{R^2}\right) \cdot (R^2 \sin(\theta) d\theta d\phi \hat{r})
\end{equation}
\end{frame}

\begin{frame}{The Dirac $\delta$-function}
(Let $d\tau$ be the volume element).  Isn't the following \textit{always} supposed to be true?
\begin{equation}
\int (\nabla \cdot \vec{v}) d\tau = \oint \vec{v} \cdot d \vec{a}
\end{equation}
We must be dealing with a strange function...apparently all of the surface integral contribution comes from the origin, where the volume element is zero, but the function is infinite. \\ \vspace{0.2cm}
Think of a function that has an finite \textit{integral} result, but is zero everywhere except one point.  Nothing comes to mind.
\end{frame}

\begin{frame}{The Dirac $\delta$-function}
The Diract $\delta$-function:
\begin{align}
\delta(x) &= 0 ~~~ if~x\neq 0 \\
\delta(x) &= \infty ~~~ if~x = 0
\end{align}
This function is called a \textit{distribution}, not a real function.  However, it has interesting properties:
\begin{align}
f(x) \delta(x) &= f(0) \delta(x) \\
\int_{-\infty}^{\infty} \delta(x) dx &= 1 \\
\int_{-\infty}^{\infty} f(x) \delta(x) dx &= f(0) \\
\int_{-\infty}^{\infty} f(x) \delta(x-a) dx &= f(a)
\end{align}
\end{frame}

\begin{frame}{The Dirac $\delta$-function}
Show that
\begin{equation}
\delta(kx) = \frac{1}{|k|}\delta(x)
\end{equation}
\end{frame}

\begin{frame}{One more interesting thing}
What is this integral?
\begin{equation}
\int_0^{2\pi} \sin(nx)\sin(mx) dx
\end{equation}
\end{frame}

\section{Conclusion}

\begin{frame}{Summary}
\begin{enumerate}
\item Electromagnetism and the module system
\begin{itemize}
\item Pace
\item Style
\item Class decision
\end{itemize}
\item Challenge level: pre-requisites
\begin{itemize}
\item Passed Calculus 1, 2, and 3
\item Passed Calculus-based physics 1, 2, and 3
\item Passed modern physics
\end{itemize}
\item Maxwell's equations live in 3D
\item \alert{Introduction to Electromagnetism by D. Griffiths (4th ed.)}
\item First half of the text is recommended by publisher, retain for graduate school
\end{enumerate}
\end{frame}

\end{document}
