\documentclass{beamer}
\usetheme{metropolis}
\usepackage{graphicx}
\usepackage{amsmath}
\title{Warm-up for Monday, February 7th, 2022}
\author{Jordan Hanson}
\institute{Whittier College Department of Physics and Astronomy}

\begin{document}
\maketitle

\section{Objects of Electromagnetism}

\begin{frame}{Objects of Electromagnetism}
What type of \textit{object} is $\vec{f}(x,y,z) \cdot \vec{g}(x,y,z)$?
\begin{itemize}
\item A: A scalar
\item B: A pseudoscalar
\item C: A vector
\item D: A pseudovector
\end{itemize}
\end{frame}

\begin{frame}{Objects of Electromagnetism}
What type of \textit{object} is $\vec{f}(x,y,z) \times \vec{g}(x,y,z)$?
\begin{itemize}
\item A: A scalar
\item B: A pseudoscalar
\item C: A vector
\item D: A pseudovector
\end{itemize}
\end{frame}

\begin{frame}{Objects of Electromagnetism}
What type of \textit{object} is $\vec{h}(x,y,z) \cdot (\vec{f}(x,y,z) \times \vec{g}(x,y,z))$?
\begin{itemize}
\item A: A scalar
\item B: A pseudoscalar
\item C: A vector
\item D: A pseudovector
\end{itemize}
\end{frame}

\begin{frame}{Objects of Electromagnetism}
What type of \textit{object} is $\nabla f(x,y,z)$?
\begin{itemize}
\item A: A scalar
\item B: A pseudoscalar
\item C: A vector
\item D: A pseudovector
\end{itemize}
\end{frame}

\begin{frame}{Objects of Electromagnetism}
What type of \textit{object} is $\frac{\partial f(x,y,z)}{\partial x}$?
\begin{itemize}
\item A: A scalar
\item B: A pseudoscalar
\item C: A vector
\item D: A pseudovector
\end{itemize}
\end{frame}

\begin{frame}{Objects of Electromagnetism}
What type of \textit{object} is $\nabla \cdot \vec{f}(x,y,z)$?
\begin{itemize}
\item A: A scalar
\item B: A pseudoscalar
\item C: A vector
\item D: A pseudovector
\end{itemize}
\end{frame}

\begin{frame}{Objects of Electromagnetism}
What type of \textit{object} is $\nabla \times \vec{f}(x,y,z)$?
\begin{itemize}
\item A: A scalar
\item B: A pseudoscalar
\item C: A vector
\item D: A pseudovector
\end{itemize}
\end{frame}

\begin{frame}{Objects of Electromagnetism}
What type of \textit{object} is $\nabla \cdot (\nabla f(x,y,z))$?
\begin{itemize}
\item A: A scalar
\item B: A pseudoscalar
\item C: A vector
\item D: A pseudovector
\end{itemize}
\end{frame}

\begin{frame}{Objects of Electromagnetism}
This object is the Laplacian of $f$:
\begin{equation}
\nabla \cdot (\nabla f(x,y,z)) = \nabla^2 f
\end{equation}
Of all the possible \textit{second derivatives} of the above objects this is the one we will encounter the most.  The rest are zero or less important (grad of divergence).  When you see a second derivative, think guilty until proven innocent, in EM.
\end{frame}

\section{Using the Gradient}

\begin{frame}{Problem 1.12}
The height of a certain hill (in meters) is given by
\begin{equation}
h(x,y) = 10(2xy -3 x^2 - 4y^2 - 18 x + 28 y + 12)
\end{equation}
where $y$ is the distance (in km) North, and $x$ is the distance East of South Hadley.
\begin{itemize}
\item Where is the top of the hill located?
\item How high is the hill?
\item How steep is the slope in meters per kilometer at a point 1 kilometer North and 1 kilometer East of South Hadley? In what direction is the slope steepest, at that point?
\end{itemize}
\end{frame}

\end{document}
