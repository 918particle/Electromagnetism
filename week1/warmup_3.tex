\documentclass{beamer}
\usetheme{metropolis}
\usepackage{graphicx}
\usepackage{amsmath}
\title{Warm-up for Monday, February 11th, 2022}
\author{Jordan Hanson}
\institute{Whittier College Department of Physics and Astronomy}

\begin{document}
\maketitle

\section{Constructing Line and Surface Integrals}

\begin{frame}{Objects of Electromagnetism}
Suppose you are performing a line integral from $x = 0$ to $x = 2$.  To what expression does $d\vec{l}$ reduce?
\begin{itemize}
\item A: $dz \hat{z}$
\item B: $dy \hat{y}$
\item C: $dx \hat{x}$
\item D: $dz \hat{x}$
\end{itemize}
\end{frame}

\begin{frame}{Objects of Electromagnetism}
Suppose you are performing a line integral from $y = 0$ to $y = 2$.  To what expression does $d\vec{l}$ reduce?
\begin{itemize}
\item A: $dz \hat{z}$
\item B: $dy \hat{y}$
\item C: $dx \hat{x}$
\item D: $dz \hat{x}$
\end{itemize}
\end{frame}

\begin{frame}{Objects of Electromagnetism}
Suppose you are performing a surface integral over the unit circle in the $xy$-plane.  To what expression does $d\vec{a}$ reduce?
\begin{itemize}
\item A: $dx dz \hat{z}$
\item B: $dx dy \hat{z}$
\item C: $s ds d\phi \hat{x}$
\item D: $s ds d\phi \hat{z}$
\end{itemize}
\end{frame}

\end{document}
