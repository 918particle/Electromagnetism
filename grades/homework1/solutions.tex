\title{Warm-up for Electromagnetic Theory (PHYS330)}
\author{Dr. Jordan Hanson - Whittier College Dept. of Physics and Astronomy}
\date{\today}
\documentclass[10pt]{article}
\usepackage[a4paper, total={18cm, 27cm}]{geometry}
\usepackage{hyperref}
\usepackage{amsmath}
\begin{document}
\maketitle

\section{Problem 1.54}

Verify the divergence theorem for $\vec{v} = r^2 \cos\theta \hat{r} + r^2 \cos\phi \hat{theta} - r^2 \cos\theta \sin\phi \hat{\phi}$ over the octant of the sphere of radius $R$ with the center at the origin. \\ \\
Break the problem into manageable pieces. (a) What is the divergence of the field? (b) What is the volume integral of it? \\ \\
Divergence: $4r\cos\theta$. \\
Volume integral of the divergence:
\begin{equation}
\int_0^R \int_0^{\pi/2} \int_0^{\pi/2} 4 r \cos\theta ~ r^2 \sin\theta dr d\theta d\phi = \frac{\pi R^4}{4}
\end{equation}
The surface integral has four parts. (a) Outer curved surface with $d\vec{a} = r^2 \sin\theta d\theta d\phi \hat{r}$, and the result is $\pi R^4/4$.  (b) The lower side is defined by $\theta = \pi/2$, and $d\vec{a} = -r^2 dr d\phi \hat{z}$.  What is $\hat{z}$ here ... $\hat{\theta}$. Consult back page of the book for conversions and set $\theta = \pi/2$.  The result is $R^4/4$. (c)  The left side is described by $\phi = 0$, and $d\vec{a} = r dr d\theta (-\hat{y}) = -r dr d\theta \hat{\phi}$.  However, the $\hat{\phi}$-component is zero for $\phi = 0$, so the surface integral is zero.  (d) The right side has $d\vec{a} = r dr d\theta \hat{\phi}$, and $\phi = \pi/2$.  This time, the surface integral for $\phi = \pi/2$ is not zero, and the result is $-R^4/4$.  Summing all the pieces, we find

\begin{equation}
\oint \vec{v} \cdot d\vec{a} = \frac{\pi R^4}{4}
\end{equation}

\section{Problem 1.55}

Break the problem into the following pieces: (a) What is the curl of $\vec{v}$?  (b) What is the surface integral of the curl?  (c) How do we approach the line integral? \\ \\

The curl may be evaluated in Cartesian coordinates: $\nabla \times \vec{v} = (b-a)\hat{k}$.  Form the surface integral:

\begin{equation}
\int (\nabla \times \vec{v}) \cdot d\vec{a} = (b-a) \pi R^2
\end{equation}

The integrand is a constant, and parallel to the area vector.  Thus, the constant moves outside the integral and we have just the area of the circle.  What is $d\vec{l}$ on the circle of radius $R$?  \textit{Cylindrical coordinates} work best to describe the situation: $d\vec{l} = ds \hat{s} + s d\phi \hat{\phi} + dz \hat{z}$.  However, $dz = 0$ and $ds = 0$, so we are left with $d\vec{l} = s d\phi \hat{\phi}$.  That makes the line integral $(s = R)$:

\begin{equation}
\oint \vec{v} \cdot d\vec{l} = \int_0^{2\pi} ay \hat{x} \cdot R d\phi \hat{\phi} + \int_0^{2\pi} bx \hat{y} \cdot R d\phi \hat{\phi} \label{eq:line1}
\end{equation}

Here are some useful conversions:
\begin{itemize}
\item $x = R\cos\phi$
\item $y = R\sin\phi$
\item $\hat{x} = 0 - \sin\phi \hat{\phi}$
\item $\hat{y} = 0 + \cos\phi \hat{\phi}$
\end{itemize}
Substituting all of that into Eq. \ref{eq:line1} gives $(b-a) \pi R^2$.

\end{document}
