\title{Warm-up for Electromagnetic Theory (PHYS330)}
\author{Dr. Jordan Hanson - Whittier College Dept. of Physics and Astronomy}
\date{\today}
\documentclass[10pt]{article}
\usepackage[a4paper, total={18cm, 27cm}]{geometry}
\usepackage{hyperref}
\usepackage{amsmath}
\begin{document}
\maketitle

\section{Problem 1.54}

Verify the divergence theorem for $\vec{v} = r^2 \cos\theta \hat{r} + r^2 \cos\phi \hat{theta} - r^2 \cos\theta \sin\phi \hat{\phi}$ over the octant of the sphere of radius $R$ with the center at the origin. \\ \\
Break the problem into manageable pieces. (a) What is the divergence of the field? (b) What is the volume integral of it? \\ \\
Divergence: $4r\cos\theta$. \\
Volume integral of the divergence:
\begin{equation}
\int_0^R \int_0^{\pi/2} \int_0^{\pi/2} 4 r \cos\theta ~ r^2 \sin\theta dr d\theta d\phi = \frac{\pi R^4}{4}
\end{equation}
The surface integral has four parts. (a) Outer curved surface with $d\vec{a} = r^2 \sin\theta d\theta d\phi \hat{r}$, and the result is $\pi R^4/4$.  (b) The lower side is defined by $\theta = \pi/2$, and $d\vec{a} = -r^2 dr d\phi \hat{z}$.  What is $\hat{z}$ here ... $\hat{\theta}$. Consult back page of the book for conversions and set $\theta = \pi/2$.  The result is $R^4/4$. (c)  The left side is described by $\phi = 0$, and $d\vec{a} = r dr d\theta (-\hat{y}) = -r dr d\theta \hat{\phi}$.  However, the $\hat{\phi}$-component is zero for $\phi = 0$, so the surface integral is zero.  (d) The right side has $d\vec{a} = r dr d\theta \hat{\phi}$, and $\phi = \pi/2$.  This time, the surface integral for $\phi = \pi/2$ is not zero, and the result is $-R^4/4$.  Summing all the pieces, we find

\begin{equation}
\oint \vec{v} \cdot d\vec{a} = \frac{\pi R^4}{4}
\end{equation}

\section{Problem 1.55}

Break the problem into the following pieces: (a) What is the curl of $\vec{v}$?  (b) What is the surface integral of the curl?  (c) How do we approach the line integral? \\ \\

The curl may be evaluated in Cartesian coordinates: $\nabla \times \vec{v} = (b-a)\hat{k}$.  Form the surface integral:

\begin{equation}
\int (\nabla \times \vec{v}) \cdot d\vec{a} = (b-a) \pi R^2
\end{equation}

The integrand is a constant, and parallel to the area vector.  Thus, the constant moves outside the integral and we have just the area of the circle.  What is $d\vec{l}$ on the circle of radius $R$?  \textit{Cylindrical coordinates} work best to describe the situation: $d\vec{l} = ds \hat{s} + s d\phi \hat{\phi} + dz \hat{z}$.  However, $dz = 0$ and $ds = 0$, so we are left with $d\vec{l} = s d\phi \hat{\phi}$.  That makes the line integral $(s = R)$:

\begin{equation}
\oint \vec{v} \cdot d\vec{l} = \int_0^{2\pi} ay \hat{x} \cdot R d\phi \hat{\phi} + \int_0^{2\pi} bx \hat{y} \cdot R d\phi \hat{\phi} \label{eq:line1}
\end{equation}

Here are some useful conversions:
\begin{itemize}
\item $x = R\cos\phi$
\item $y = R\sin\phi$
\item $\hat{x} = 0 - \sin\phi \hat{\phi}$
\item $\hat{y} = 0 + \cos\phi \hat{\phi}$
\end{itemize}
Substituting all of that into Eq. \ref{eq:line1} gives $(b-a) \pi R^2$.

\clearpage

\section{Problem 1.56}

Break the problem into pieces.  First, address the closed-path line integral.  For the first path,

\begin{align}
d\vec{l} &= dy \hat{y} \\
\vec{v} \cdot d\vec{l} &= y z^2 dy \\
z &= 0 \\
\int \vec{v} \cdot d\vec{l} &= 0
\end{align}

For the other straight piece,

\begin{align}
d\vec{l} &= dz \hat{z} \\
\vec{v} \cdot d\vec{l} &= (3y + z) dz \\
x = y &= 0 \\
\int \vec{v} \cdot d\vec{l} &= -\int_0^{2} (3y+z) dz = -2
\end{align}

For the diagonal piece, the path has $x = 0$, and $z = 2 - 2y$, with $dz = - 2 dy$.  We have

\begin{align}
d\vec{l} &= dy \hat{y} + dz \hat{z} \\
\int \vec{v} \cdot d\vec{l} &= \int_{1}^{0} dy \left( 4 y^3 - 8y^2 +2 y - 4\right) = \frac{14}{3} \\
\end{align}

In total, the close-path line integral is $\boxed{-2 + 14/3 = 8/3}$.  The curl of the field is

\begin{equation}
\nabla \times \vec{v} = (3 - 2 y z)\hat{x} + ...
\end{equation}

\textit{We don't need the other components of the curl because the area vector will just cancel them:} $d\vec{a} = dy dz \hat{x}$.

\begin{align}
\int (\nabla \times \vec{v}) \cdot d\vec{a} &= \int_{0}^{1} \int_{0}^{2 - 2y} dz dy (3 - 2 y z) \label{eq:limit} \\
\int_{0}^{1} dy \left( 4y^3 - 8y^2 + 10y -6 \right) &= \frac{8}{3}
\end{align}

Notice in Eq. \ref{eq:limit} that we integrate $z$ from $0$ to $z_{max}$, where $z_{max}$ is determined by the relationship between $z$ and $y$.  Thus, Stoke's theorem checks out.

\section{Problem 1.57}

This exercise helps us practice with coordinate systems besides Cartesian.  The line integral involves four pieces.  The first is in the $x$-direction.  In spherical coordinates:

\begin{align}
d\vec{l} &= dr \hat{r} \\
\phi &= 0, ~ \theta = \pi/2 \\
\vec{v} \cdot d\vec{l} &= r \cos^2\theta ~ dr = 0
\end{align}

The second piece is in the $xy$-plane, with $\theta = \pi/2$ anmd $r = 1$.  In spherical coordinates:

\begin{align}
d\vec{l} &= d\phi \hat{\phi} \\
\phi &= 0, ~ \theta = \pi/2 \\
\vec{v} \cdot d\vec{l} &= r \cos^2\theta ~ dr = 0
\end{align}

The result is

\begin{equation}
\int \vec{v} \cdot d\vec{l} = \int_{0}^{3\pi/2} 3 r d\phi = \frac{3\pi}{2}
\end{equation}

The third piece is in the $z$-direction, with $y = 1$ and $x = 0$.  We have

\begin{align}
d\vec{l} &= dr \hat{r} + r d\theta \hat{\theta} \\
\vec{v} \cdot d\vec{l} &= r \cos^2\theta dr - r^2 \cos\theta\sin\theta d\theta \\
y &= r\sin\theta = 1 ~~ (r = 1/\sin\theta) \\
dr &= -\frac{\cos\theta}{\sin^2\theta} d\theta \\
\vec{v} \cdot d\vec{l} &= (-\cot^3\theta - \cot\theta) d\theta
\end{align}

The line integral can therefore be cast in terms of $\theta$ only, and integrated from $\theta = \pi/2$ to $\tan^{-1}(1/2)$.  The result is 

\begin{equation}
\int \vec{v} \cdot d\vec{l} = -\frac{1}{2} \left. \frac{1}{\sin^2\theta} \right|_{\pi/2}^{\tan^{-1}(1/2)} = 2
\end{equation}

For the last piece, the path is along $r$, while $\phi = \pi/2$ \textit{and} $\theta = \theta_0 = \tan^{-1}(1/2)$ remain fixed.  We find

\begin{align}
d\vec{l} &= dr \hat{r} \\
\vec{v} \cdot d\vec{l} &= \cos^2\theta_0 r dr \\
\int \vec{v} \cdot d\vec{l} &= \cos^2\theta_0 \int_{\sqrt{5}}^{0} r dr = -2
\end{align}

Totaling the four contributions to the line integral: $\boxed{3\pi/2 + 2 - 2 = 3\pi/2}$.  Checking Stoke's theorem requires the curl in spherical coordinates:

\begin{equation}
\nabla \times \vec{v} = 3 \cot\theta ~ \hat{r} - 6 \hat{\theta}
\end{equation}

The surface integral of the bottom face is ($d\vec{a} = -r dr d\phi \hat{\theta}$):

\begin{equation}
\int \nabla \times \vec{v} \cdot d\vec{a} = \int_0^{\pi/2} \int_0^{1} 6 r dr d\phi = \frac{3\pi}{2}
\end{equation}

For the back face, $d\vec{a} = da \hat{\phi}$.  But the curl does not have a $\hat{\phi}$-component, so that surface integral is zero. Thus, Stoke's Theorem checks out.

\section{Problem 1.59}

First, find the divergence using spherical coordinates:

\begin{equation}
\nabla \cdot \vec{v} = 4 r \cot\theta\cos\theta
\end{equation}

Integrate over the slice of the sphere with radius $R$ and opening angle $\theta = \pi/6$.

\begin{equation}
\int_0^{R} \int_0^{2\pi} \int_0^{\pi/6} 4 r \cot\theta\cos\theta ~ r^2 \sin\theta dr d\theta d\phi = 2\pi R^4 \int_0^{\pi/6} \cos^2\theta d\theta = \boxed{\frac{\pi R^4}{12}(2\pi + 3\sqrt{3})}
\end{equation}

The closed surface integral must be broken into the ``cone'' portion, and the ``top'' portion.  For the top, we have

\begin{align}
d\vec{a} &= R^2 \sin\theta d\theta d\phi \hat{r} \\
\vec{v} \cdot d\vec{a} &= R^4 \sin^2\theta d\theta d\phi \\
\int \vec{v} \cdot d\vec{a} &= 2\pi R^4 \int_{0}^{\pi/6} \sin^2\theta d\theta \\
\int \vec{v} \cdot d\vec{a} &= \frac{\pi R^4}{12}(2\pi - 3\sqrt{3})
\end{align}

For the cone portion:

\begin{align}
d\vec{a} &= \frac{1}{2} r dr d\theta d\phi \hat{\theta} \\
\int \vec{v} \cdot d\vec{a} &= \int_0^{1} \int_{0}^{2\pi} \sqrt{3} r^3 dr d\phi = \frac{\pi \sqrt{3} R^4}{2} 
\end{align}

Summing the top and the cone, we find the surface integral total is $\boxed{\frac{\pi R^4}{12}(2\pi + 3\sqrt{3})}$.

\section{Problem 1.62}

\begin{itemize}

\item (a) First, note that $d\vec{a} = R^2 \sin\theta d\theta d\phi \hat{r}$.  Integrating just $d\vec{a}$ should yield a vector, which can be broken into x, y, and z-components.  By symmetry, there should be no x or y-components.  Just the z-component of $\hat{r}$ is $\cos\theta \hat{z}$ (back cover of the textbook). Integrating:

\begin{equation}
\vec{a} = 2 \pi R^2 \hat{z} \int_{0}^{\pi/2} \sin\theta\cos\theta ~ d\theta = \pi R^2 \hat{z}
\end{equation}

In other words, we find the projected cross-sectional area, that of a circle and not of a hemisphere.

\item (b) Note that Problem 1.61 (a) says that

\begin{equation}
\int_{\mathcal{V}} (\nabla T) d\tau = \oint_{\mathcal{S}} T d\vec{a}
\end{equation}

This is the type of formula that follows from the other fundamental theorems of calculus.  It says that the volume integral over a vector field that is the gradient of a scalar is equal to the closed surface integral of the scalar.  However, we can let $T(x,y,z) = 1$ so that the right hand side is 

\begin{equation}
\oint_{\mathcal{S}} d\vec{a} = \int_{\mathcal{V}} (\nabla ~ 1) d\tau = 0
\end{equation}

Thus, all closed surface integrals of constants are zero.

\item (c)  Suppose there are two surfaces $\mathcal{S}_1$ and $\mathcal{S}_2$ that share the same boundary line.  Adding the surface integrals:

\begin{equation}
\oint_{\mathcal{S}_1} d\vec{a} + \oint_{\mathcal{S}_2} d\vec{a} = \vec{a}_{\rm total}
\end{equation}

But the two surfaces now form a closed surface, so $\vec{a}_{\rm total} = \vec{0}$ (part b).  Further, the normal directions of $\mathcal{S}_1$ and $\mathcal{S}_2$ differ by a minus sign, so we find

\begin{align}
\oint_{\mathcal{S}_1} d\vec{a} - \oint_{\mathcal{S}_2} d\vec{a} &= 0 \\
\oint_{\mathcal{S}_1} d\vec{a} &= \oint_{\mathcal{S}_2} d\vec{a}
\end{align}

\item (d) For the kind of triangle described in the hint, $d\vec{a} = \frac{1}{2}\vec{r} \times d\vec{l}$, since the cross product can be interpreted as the area of a parallelogram and we need one half of that parallelogram.  Totalling all of the triangles around the surface:

\begin{equation}
\vec{a} = \oint d\vec{a} = \oint \frac{1}{2}\vec{r} \times d\vec{l}
\end{equation}

\item (e) Letting $T = \vec{c} \cdot \vec{r}$ in 1.61 (e), we find

\begin{equation}
- \oint (\vec{c} \cdot \vec{r}) d\vec{l} =  \int_{\mathcal{S}} \nabla (\vec{c} \cdot \vec{r}) \times d\vec{a}
\end{equation}

From the reading, we need a product rule for the gradient on the left side:

\begin{align}
\nabla (\vec{c} \cdot \vec{r}) &= \vec{c} \times (\nabla \times \vec{r}) + (\vec{c} \cdot \nabla)\vec{r} \\
\nabla (\vec{c} \cdot \vec{r}) &= (\vec{c} \cdot \nabla)\vec{r} = \vec{c} \\
(\nabla \times \vec{r} &= 0)
\end{align}

Using that result gives

\begin{equation}
\oint (\vec{c} \cdot \vec{r}) d\vec{l} =  -\int_{\mathcal{S}} \vec{c} \times d\vec{a} = - \vec{c} \times \vec{a} = \vec{a} \times \vec{c}
\end{equation}

Reversing the order of the cross-product removes the minus sign in the final step.

\begin{equation}
\vec{a} \times \vec{c} = \oint (\vec{c} \cdot \vec{r}) d\vec{l}
\end{equation}
\end{itemize}

\section{Problem 1.63}

\end{document}
