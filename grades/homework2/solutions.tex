\title{Solutions for Homework 2}
\author{Dr. Jordan Hanson - Whittier College Dept. of Physics and Astronomy}
\date{\today}
\documentclass[10pt]{article}
\usepackage[a4paper, total={18cm, 27cm}]{geometry}
\usepackage{graphicx}
\usepackage{amsmath}
\usepackage{tcolorbox}

\def\rcurs{{\mbox{$\resizebox{.16in}{.08in}{\includegraphics{ScriptR}}$}}}
\def\brcurs{{\mbox{$\resizebox{.16in}{.08in}{\includegraphics{BoldR}}$}}}
\def\hrcurs{{\mbox{$\hat \rcurs$}}}

\begin{document}
\maketitle

\section{Problem 2.5}

\textit{Find the electric field a distance $z$ above the center of a circular loop of radius $r$ that carries a uniform line charge $\lambda$.} \\ \\

Start by filling in the pieces of the Coulomb effect:

\begin{equation}
d\mathbf{E} = \frac{1}{4\pi\epsilon_0}\frac{dq'}{\rcurs^2}\hrcurs
\end{equation}

\begin{itemize}
\item $\brcurs = z \hat{z} - R\hat{s}$
\item $\rcurs = \sqrt{z^2 + R^2} = z\sqrt{1+(R/z)^2} = z\sqrt{1+\epsilon^2}$. (If $\epsilon \to 0$, this represents the far-field).
\item $\rcurs^2 = z^2 + R^2$
\item $\hrcurs = (\hat{z} - \epsilon \hat{s})/(1+\epsilon^2)^{1/2}$
\item $dq' = \lambda R d\phi'$, because cylindrical coordinates are the correct choice here.
\item Note that $\epsilon = R/z$, so if $z = 0$ then $\epsilon \to \infty$, and $\epsilon \to 0$ if $z \gg R$.
\item $Q = 2\pi R\lambda$, the total charge.
\end{itemize}

Integrate to add up the $d\mathbf{E}$ to find $\mathbf{E}$:

\begin{equation}
\mathbf{E} = \frac{\lambda R}{4\pi\epsilon_0 z^2 (1+\epsilon^2)^{3/2}} \int_0^{2\pi} d\phi' (\hat{z} - \epsilon \hat{s})
\end{equation}

By symmetry, the $\hat{s}$ term evaluates to zero.  The result is

\begin{equation}
\mathbf{E} = \frac{2\pi\lambda R \hat{z}}{4\pi\epsilon_0 z^2 (1+\epsilon^2)^{3/2}} = \frac{Q\hat{z}}{4\pi\epsilon_0 z^2 (1+\epsilon^2)^{3/2}}
\end{equation}

Checks: $\mathbf{E} = 0$ if $z = 0$ because $\epsilon \to \infty$.  Also, we find the far-field effect if $z \ll R$, because $\epsilon \to 0$.  The units also check out.

\section{Problem 2.6}

\textit{Find the electric field a distance $z$ above a the center of a flat circular disc of radius $R$ that carries a uniform surface charge $\sigma$.  What does your formula give in the limit $R \to \infty$? Also check the case $z \gg R$.} \\ \\

Start by filling in the pieces of the Coulomb effect:

\begin{equation}
d\mathbf{E} = \frac{1}{4\pi\epsilon_0}\frac{dq'}{\rcurs^2}\hrcurs
\end{equation}

\begin{itemize}
\item $\brcurs = z\hat{z} - s\hat{s}$.  As in Problem 2.5, the $\hat{s}$-component will vanish upon integration.
\item $\rcurs^2 = z^2 + s^2$
\item $\hrcurs = (z\hat{z} - s\hat{s})/(z^2 + s^2)^{1/2}$
\item $dq' = \sigma da' = s ds d\phi$, because cylindrical coordinates work best here.
\item $Q = \pi R^2 \sigma$ is the total charge.
\item Let $z\tan\theta = s$, so that $ds = z\sec^2\theta d\theta$, and $\theta_0 = \tan^{-1}(R/z)$
\end{itemize}

Integrate to add up the $d\mathbf{E}$ to find $\mathbf{E}$:

\begin{align}
\mathbf{E} &= \frac{\sigma}{2\epsilon_0} z \hat{z} \int_0^R \frac{s ds}{(s^2+z^2)^{3/2}} \\
\mathbf{E} &= \frac{\sigma}{2\epsilon_0} \left.\cos\theta \right\vert_0^{\theta_0} \hat{z} = \frac{\sigma}{2\epsilon_0} \left( 1 - \cos\theta_0 \right) \hat{z}
\end{align}

We know what is $\tan\theta = R/z$, but what is $\cos\theta_0$?  (\textit{Hint: draw the triangle and then find the hypoteneuse}).  The result is

\begin{equation}
\mathbf{E} = \frac{\sigma \hat{z}}{2\epsilon_0}\left( 1 - \frac{z}{\sqrt{z^2 + R^2}}\right)
\end{equation}

The units check automatically because of the units of $\sigma$ (Coulombs per meter squared), and the $\epsilon_0$ in the denominator.  If $R \to \infty$, then the second term vanishes and the field is 

\begin{equation}
\mathbf{E} \to \frac{\sigma \hat{z}}{2\epsilon_0}
\end{equation}

This form of the field is the boundary condition near a charged surface.  To check the limit that $z \gg R$, first factor a $z$:

\begin{equation}
\mathbf{E} = \frac{\sigma \hat{z}}{2\epsilon_0} z \left( z^{-1} - (z^2 + R^2)^{-1/2}\right) = \frac{\sigma \hat{z}}{2\epsilon_0} z \left( z^{-1} - z^{-1} (1 + (R/z)^2)^{-1/2}\right)
\end{equation}

Now replace $(1+(R/z)^2)^{-1/2} \approx (1 - (1/2) (R/z)^2)$, and notice the $1/z$ terms cancel:

\begin{equation}
\mathbf{E} = \frac{\sigma \hat{z}}{2\epsilon_0}z\left( \frac{1}{2z} \left(\frac{R}{z}\right)^2 \right)
\end{equation}

Multiply top and bottom by $\pi$, and recall that $Q = \pi R^2$ to find the point-charge field:

\begin{equation}
\mathbf{E} = \frac{1}{4\pi\epsilon_0} \frac{Q \hat{z}}{z^2}
\end{equation}

\section{Problem 2.9}

\textit{Suppose the electric field in some region is found to be $\mathbf{E} = k r^3 \hat{r}$, in spherical coordinates (k is some constant). (a) Find the charge densiy $\rho$. (b) Find the total charge contained in a sphere of radius $R$, centered at the origin. (Do it two ways).} \\ \\

(a) This is a straightforward application of Gauss' Law in differential form, noting that only the $\hat{r}$ term matters by symmetry:

\begin{equation}
\nabla \cdot \mathbf{E} = \frac{1}{r^2} \frac{\partial}{\partial r}\left( r^2 \mathbf{E} \cdot \hat{r} \right) + ... = 5kr^2
\end{equation} 

Thus the charge distribution is 

\begin{equation}
\rho(r) = 5 k \epsilon_0 r^2 \label{eq:rho}
\end{equation}

(b) The total charge $Q$ may be found by (i) direct integration of Eq. \ref{eq:rho}, or (ii) by using Gauss' Law.  First, (i):

\begin{equation}
Q = \int \rho(r) d\tau = \int_0^{R} \int_0^{\pi} \int_0^{2\pi} 5 k \epsilon_0 r^2 r^2 \sin\theta dr d\theta d\phi = (4\pi)(5k\epsilon_0) \int_0^{R} r^4 dr = 4\pi k \epsilon_0 R^5
\end{equation}

Now, method (ii): 

\begin{equation}
Q = \epsilon_0 \oint \mathbf{E} \cdot d\mathbf{a} = \epsilon_0 \int_0^{2\pi} \int_0^{\pi} k R^3 \hat{r} \cdot \hat{r} R^2 d\theta d\phi = 4\pi k \epsilon_0 R^5
\end{equation}

\section{Problem 2.12}

\textit{Use Gauss' Law to find the electric field inside a uniformly charged solid sphere (charge density $\rho$).} \\ \\

Note that

\begin{itemize}
\item $d\mathbf{a} \propto \hat{r}$
\item $\mathbf{E} \propto \hat{r}$ and only varies with $r$, by symmetry.
\item Along a spherical Gaussian surface at radius $r$, $\mathbf{E}$ is constant.
\item Thus, $\oint \mathbf{E} \cdot d\mathbf{a} = \mathbf{E} \cdot \mathbf{A}$, where $\mathbf{A}$ is the surface area of the Gaussian surface oriented in the $\hat{r}$ direction.
\end{itemize}

We now have

\begin{equation}
\mathbf{E} \cdot \mathbf{A} = \frac{1}{\epsilon_0} \rho \int d\tau' = \frac{\rho}{\epsilon_0} (4\pi) \int_0^{r} r'^2 dr'
\end{equation}

The final result is

\begin{equation}
\mathbf{E}(\mathbf{r}) = \frac{\rho \mathbf{r}}{3\epsilon_0} = \frac{\rho r\hat{r}}{3\epsilon_0}
\end{equation}

\end{document}
