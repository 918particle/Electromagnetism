\title{Solutions for Homework 4}
\author{Dr. Jordan Hanson - Whittier College Dept. of Physics and Astronomy}
\date{\today}
\documentclass[10pt]{article}
\usepackage[a4paper, total={18cm, 27cm}]{geometry}
\usepackage{graphicx}
\usepackage{amsmath}
\usepackage{tcolorbox}

\def\rcurs{{\mbox{$\resizebox{.16in}{.08in}{\includegraphics{ScriptR}}$}}}
\def\brcurs{{\mbox{$\resizebox{.16in}{.08in}{\includegraphics{BoldR}}$}}}
\def\hrcurs{{\mbox{$\hat \rcurs$}}}

\begin{document}
\maketitle

\section{Problem 4.10}

\textit{A sphere of radius $R$ carries a polarization}
\begin{equation}
\mathbf{P}(\mathbf{r}) = k\mathbf{r} \label{eq:P}
\end{equation}
\textit{In Eq. \ref{eq:P}, $k$ is a constant and and $\mathbf{r}$ is the vector from the center.
\begin{itemize}
\item (a) Calculate the bound charges $\sigma_b$ and $\rho_b$.
\item (b) Find the field inside and outside the sphere.
\end{itemize}
}
(a) The surface bound charge is at radius $R$, so $\sigma_b = \mathbf{P} \cdot \hat{\mathbf{n}} = kR$.  The volumetric bound charge is $\rho_b = - \nabla \cdot \mathbf{P} = -3k$.  Note the total charge should add up to zero: $(4\pi R^2) \sigma_b + (4/3)\pi R^3 \rho_b = 4\pi R^3 k - 4\pi R^3 k = 0$. (b) The field of any constant volumetric charge density $-3k$ should be calculable via Gauss' law.  We find, after integrating over a Gaussian surface of radius $r<R$:
\begin{align}
\oint \mathbf{E} \cdot d\mathbf{a} = \mathbf{E} \cdot \mathbf{A} &= \frac{1}{\epsilon_0}\rho = \frac{-4k\pi r^3}{\epsilon_0} \\
\mathbf{E} &= -\frac{3kr}{\epsilon_0} \hat{\mathbf{r}} = -\mathbf{P}/\epsilon_0
\end{align}
(b) Note that, because the net charge is zero, the field outside the sphere is zero.

\section{Problem 4.14}

\textit{When you polarize a neutral dielectric, the charge moves a bit, but the total remains zero.  This fact should be reflected in the bound charges $\sigma_b$ and $\rho_b$.  Prove from Eqs. 4.11 and 4.12 that the toal bound charge vanishes.} \\

First, let's integrate the total volumetric bound charge:

\begin{equation}
-q = \int_{\mathcal{V}} \rho_b d\tau = -\int_{\mathcal{V}} \nabla \cdot \mathbf{P} d\tau = -\oint \mathbf{P} \cdot \hat{\mathbf{n}} da = -\oint \sigma_b da
\end{equation}

Next, the total surface bound charge is

\begin{equation}
q = \oint_S \sigma_b da
\end{equation}

Now we see that $Q = -q + q = 0$.

\section{Problem 4.15}

\textit{A thick spherical shell (inner radius a, outer radius b) is made of dielectric material with a frozen-in polarization}

\begin{equation}
\mathbf{P}(\mathbf{r}) = \frac{k}{r}\hat{\mathbf{r}} \label{eq:P2}
\end{equation}

In Eq. \ref{eq:P2}, $k$ is a constant and $r$ is the distance from the center.  There is no free charge in the problem.  Find the electric field in all three regions by two different methods:
\begin{itemize}
\item (a) Locate all the bound charge, and use Gauss' Law to calculate the field it produces.
\item (b) Use Eq. 4.23 to find $\mathbf{D}$, and then get $\mathbf{E}$ from Eq. 4.21. [Notice the second method is faster, and it avoids any explicit reference to the bound charges].
\end{itemize}

\end{document}
