\title{Midterm for Electromagnetic Theory (PHYS330)}
\author{Dr. Jordan Hanson - Whittier College Dept. of Physics and Astronomy}
\date{\today}
\documentclass[10pt]{article}
\usepackage[a4paper, total={18cm, 27cm}]{geometry}
\usepackage{outlines}
\usepackage{hyperref}
\begin{document}
\maketitle

\begin{abstract}
This exam may be completed at home, and covers chapters 1-4 of the course text and in-class examples.  Class notes and the course text may be used (open book), but no internet sources are allowed.  The daily warm-up exercises are good study materials for this exam.
\end{abstract}
\noindent

\section{Math Bootcamp}

\begin{enumerate}
\item (a) If $\mathbf{A}$ and $\mathbf{B}$ are two vector functions, what does the expression $(\mathbf{A} \cdot \nabla) \mathbf{B}$ mean?  That is, what are its $x$, $y$, and $z$ components, in terms of the Cartesian components of $\mathbf{A}$, $\nabla$, and $\mathbf{B}$? (b) Compute $(\hat{r} \cdot \nabla) \hat{r}$, where $\hat{r}$ is $\mathbf{r}/r$. (c) One can show that the \textit{force} on a dipole induced by a non-uniform field is
\begin{equation}
\mathbf{F} = (\mathbf{p} \cdot \nabla) \mathbf{E}
\end{equation}
Compute the force on a physical dipole located at the origin with $\mathbf{p} = q \mathbf{d} = qd~\mathbf{\hat{x}}$ in a field with associated potential $V(\mathbf{r}) = V_0 r^2 + V_1$. \\ \vspace{3cm}
\item Evaluate the following integral using (a) the three-dimensional Dirac delta function, or (b) integration by parts.  Solving both earns a bonus point.
\begin{equation}
J = \int_{\mathcal{V}} e^{-r} \left( \nabla \cdot \frac{\mathbf{\hat{r}}}{r^2} \right)
\end{equation} \\ \vspace{2.5cm}
\end{enumerate}

\section{Electrostatics}

\begin{enumerate}
\item Suppose two dipoles, each with dipole moment $\mathbf{p}$ pointed in opposite directions, form a square with alternating positive and negative charges and side length $d$.  Calculate the field $\mathbf{E}_{\rm tot}$ at the following points $P$: (a) $P = (0,0)$, (b) $P = (2d,0)$, and $P = (0,2d)$.  Check units and take limits\footnote{This object is an electrostatic quadrupole.}.  \\ \vspace{3cm}
\item The electric potential of some configuration is given by the expression
\begin{equation}
V(\mathbf{r}) = A \frac{e^{-\lambda r}}{r}
\end{equation}
\end{enumerate}

\end{document}
