\title{Reading Quiz 2 for Electromagnetic Theory (PHYS330)}
\author{Dr. Jordan Hanson - Whittier College Dept. of Physics and Astronomy}
\date{\today}
\documentclass[10pt]{article}
\usepackage[a4paper, total={18cm, 27cm}]{geometry}
\usepackage{outlines}
\usepackage{hyperref}
\begin{document}
\maketitle

\begin{abstract}
A summary of content covered in chapter 2 of Introduction to Electrodynamics. 
\end{abstract}
\noindent

\section{Distributions of Point Charges}

\begin{enumerate}
\item Picture a \textit{physical dipole} of two charges $+q$ and $-q$ of equal magnitude, separated by a distance $d$.  Define the dipole moment as $\vec{p} = q\vec{d}$ pointing from $-q$ to $q$ somewhere in the xy-plane.  Now add an external electric field $\vec{E} = E_0 \hat{x}$.  Show that the \textit{torque} on the dipole is 
\begin{equation}
\vec{\tau} = \vec{p} \times \vec{E}
\end{equation} \\ \vspace{3cm}
\item Imagine two dipoles, each with dipole moments $\vec{p}_1$ and $\vec{p}_2$ pointed in opposite directions, forming a square with alternating positive and negative charges.  Calculate the electric field vector in the center of the square. \\ \vspace{3cm}
\end{enumerate}

\section{Continuous Charge Distributions}

\begin{enumerate}
\item (a) Compute the electric field of a continuous line of charge, with total charge $Q = \lambda L$, where $\lambda$ is the charge density and $L$ is the total length.  Take the field point to be a distance $z$ above the center of the line of charge.  Show what happens in the limit that $L \gg z$. (b) Obtain the same result as (a) using Gauss' Law. \\ \vspace{5cm}
\item Assuming a plane of charge with charge density (Coulombs per unit area) $\sigma$ has an electric field $\sigma/(2\epsilon_0)$, what electric fields would occur in each of the following situations:
\begin{itemize}
\item Two planes of positive charge, and the field point is somewhere between the plates.
\item Two planes of charge, one positive and one negative, and the field point is somewhere between the plates.
\item Two planes of positive charge, one occupying the yz-plane, and the other occupying the xz-plane, and the field point is $(1,1,0)$.
\end{itemize}
\vspace{3cm}
\end{enumerate}

\section{The Curl of $\vec{E}$-fields}

\begin{enumerate}
\item According to Eq. 2.19 in the text, the close loop line integral for the E-field of a point charge is 
\begin{equation}
\oint \vec{E} \cdot d\vec{l} = 0
\end{equation}
This implies $\nabla \times \vec{E} = 0$.  According to the Helmholtz theorem in Ch. 1, this means the $\vec{E}$-field can be cast as the gradient of a scalar function known as \textit{the potential, $V$}:
\begin{equation}
\vec{E} = -\nabla V
\end{equation}
The minus sign is a convention that is analogous to the minus sign in $\vec{F} = - \frac{dU}{dx} \hat{x}$ from mechanics.
\begin{itemize}
\item Show that 
\begin{equation}
- \int_a^{B} \vec{E} \cdot d\vec{l} = V(\vec{b}) - V(\vec{a})
\end{equation}
\item Assume a point charge at the origin, and label its electric field $\vec{E}$.  Perform the integral
\begin{equation}
V(\vec{r}) = - \int_{\infty}^{r} E(r') dr'
\end{equation}
to find the potential formula for a point charge. \textit{[Answer: $kq/r$]}
\end{itemize}
\end{enumerate}

\end{document}
