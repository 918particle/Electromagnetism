\title{Warm-Up for April 8th, 2022}
\author{Dr. Jordan Hanson - Whittier College Dept. of Physics and Astronomy}
\date{\today}
\documentclass[12pt]{article}
\usepackage[a4paper, total={18cm, 27cm}]{geometry}
\usepackage{graphicx}
\usepackage{amsmath}
\usepackage{bm}
\def\rcurs{{\mbox{$\resizebox{.16in}{.08in}{\includegraphics{ScriptR}}$}}}
\def\brcurs{{\mbox{$\resizebox{.16in}{.08in}{\includegraphics{BoldR}}$}}}
\def\hrcurs{{\mbox{$\hat \brcurs$}}}
 
\begin{document}
\maketitle
\small
\section{Memory Bank}
\begin{enumerate}
\item Lorentz Force ... $\mathbf{F} = I \mathbf{L} \times \mathbf{B}$
\item Torque ... $\boldsymbol\tau = \mathbf{r} \times \mathbf{F}$
\item Continuity Equation ... $\nabla \cdot \mathbf{J} = -\frac{\partial \rho}{\partial t}$
\end{enumerate}

\section{Torque on a Current Loop, Current Density}

\begin{enumerate}
\item Suppose there is a square loop of current in the $xy$-plane, with side length $a$, centered on the origin.  There is a constant, uniform magnetic field $\mathbf{B} = B \hat{\mathbf{x}}$.  Let $\boldsymbol\mu = I \mathbf{A}$, where $\mathbf{A}$ is an area vector for the loop.  Show that there is a torque on the loop, equal to $\boldsymbol\tau = \boldsymbol\mu \times \mathbf{B}$. \\ \vspace{5cm}
\item Suppose the outflow of charged particles from some exploding astrophysical event has spherical symmetry and decays exponentially:
\begin{equation}
\mathbf{J} = I_0 \left(\frac{e^{-\lambda t}}{r^2}\right)\hat{\mathbf{r}}
\end{equation}
Using the continuity equation, show that the charge lost by the exploding object is 
\begin{equation}
Q(t) = \frac{4\pi I_0}{\lambda}\left(e^{-1} - 1\right)
\end{equation}
\end{enumerate}

\end{document}