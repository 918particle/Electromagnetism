\title{Syllabus for Electromagnetic Theory (PHYS330)}
\author{Dr. Jordan Hanson - Whittier College Dept. of Physics and Astronomy}
\date{\today}
\documentclass[10pt]{article}
\usepackage[a4paper, total={18cm, 27cm}]{geometry}
\usepackage{outlines}
\usepackage{hyperref}
\begin{document}
\maketitle

\begin{abstract}
Maxwell's Equations govern the electromagnetic fields that constitute electrodynamics.  However, an understanding of Maxwell's equations must be built from the fundamental building blocks of vector calculus, bound and free electric charge, electrostatics and electric potential, magnetostatics and magnetic vector potential, and the continuity of free and bound current.  These building blocks will be presented in the traditional order, as shown in \textit{Introduction to Electrodynamics} by D. Griffiths.  Although the course culimates with the presentation of Maxwell's Equations, special topics will be covered along the way.  These include multipole expansions, Fourier transforms and Fourier analysis, and numerical simulation methods.  
\end{abstract}
\noindent
\textit{\textbf{Pre-requisites}:  Calculus-based Physics 1-3, Calculus 1-3, and Modern Physics.} \\
\textit{\textbf{Course credits, Liberal Arts Categorization}: 3.0 Credits, None} \\
\textit{\textbf{Regular course hours}: MWF 10:00-10:50.  \textit{Zoom ID/pass}: 796 092 0745 / 667725. } \\
\textit{\textbf{Instructor contact information}: Email: jhanson2@whittier.edu, Zoom ID / pass: 796 092 0745 / 667725, YouTube Channel: \url{www.youtube.com/918particle}, Book online appointments: \url{https://fgucmvjkylvmgqfsco.10to8.com}}
\textit{\textbf{Office hours}: Please use the booking service to schedule Zoom meetings for office hours and advising: \url{10to8.com} as above.} \\
\textit{\textbf{Attendance/Absence}: As this is an advanced course, there is no attendance policy. Students will hold themselves accountable, and the chances of passing the course increase substantially with regular class attendance.} \\ 
\textit{\textbf{Late work policy}: Late work is generally not accepted, but is left to the discretion of the instructor.} \\
\textit{\textbf{Text}: Introduction to Electrodynamics, 4th ed. by D. Griffiths - \url{https://en.wikipedia.org/wiki/Introduction_to_Electrodynamics}}. \\
\textit{\textbf{Grading}: There will be weekly homework sets worth 50 percent of the grade, typically due Mondays.  Homework problems are challenging in this subject, and some collaboration is expected.  There will be in-class warm-up problems that will be graded for completion, and worth 10 percent of the grade.  There will be one midterm exam worth 20 percent of the grade.  There will be no final exam for the course.  Instead, there will be a final project, plus a presentation to the class.  The project will be worth 20 percent of the grade.  See Tab. \ref{tab:grading}.}
\begin{table}
\centering
\begin{tabular}{| c | c | c |}
\hline
Item & Fraction of Grade & Note \\ \hline
Homework & 50\% & Submit in PDF Form to Moodle \\ \hline
Warm Ups & 10\% & Graded for Completion \\ \hline
Midterm Exam & 20\% & Take-home style, open-book \\ \hline
Final Project and Presentation & 20\% & Live or digital (see \textbf{Final Projects} above). \\ \hline
\end{tabular}
\caption{\label{tab:grading} Grading percentages for PHYS330.}
\end{table} \\
\textit{\textbf{Grade Settings}: $<60\%$ = F, $\geq 60\%, <70\%$ = D, $\geq 70\%, <80\%$ = C, $\geq 80\%, <90\%$ = B, $\geq 90\%, <100\%$ = A.  Pluses and minuses: 0-3\% minus, 3\%-6\% straight, 6\%-10\% plus (e.g. 79\% = C+, 91\% = A-)} \\
\textit{\textbf{Homework Sets}: Five to ten problems per week, assigned and collected on Fridays via Moodle PDF submission.} \\
\textit{\textbf{Final projects}: The final project will involve one of two types of topic: A) a more lengthy or challenging problem to teach the class, or B) a computational electromagnetism (CEM) example.  The final presentation should be longer, approximately 20 minutes.  Students should begin gathering and/or sharing data early on in the course to facilitate creation of the final presentation.  The presentation can be in traditional format or as a digital storytelling project.  For digital storytelling projects, Whittier College has a site license for the online video editing package WeVideo.  A short WeVideo tutorial will be given as part of the course.} \\
\textit{\textbf{Statement on Disability Services}: Whittier College is committed to make learning experiences as accessible as possible. If you experience physical or academic barriers due to a disability, you are encouraged to contact Student Disability Services (SDS) to discuss options. To learn more about academic accommodations, please email disabilityservices@whittier.edu.} \\
\textit{\textbf{Mental Health Resources:} Counseling services for enrolled students are offered at no charge and will be offered remotely during periods of remote learning. When we return to campus, in person and remote/telehealth services will be offered. Schedule an appointment by emailing counselingcenter@whittier.edu  or by phone, 562-907-4239 (between 8am – 5pm; M-F). For support after 5pm, you may call the After-Hours RN Telephone Advice Line at 562.454.4548 (press option 1); for mental health emergencies contact Campus Safety at 562-907-4211; Digital mental health platform at \url{https://you.whittier.edu/}} \\ \\
\textit{\textbf{Changes due to COVID-19}:
\begin{enumerate}
\item Class will meet daily using the Zoom web-conferencing until February 21st, 2022.  Class time will be used to review course status, homework, student questions, and work examples.  Content will also be delivered via the reading and video tutorials on the instructor YouTube channel.
\item It is critical that students take advantages of one-on-one meetings with the professor.  Please book an appointment time via 10to8.com: \url{https://fgucmvjkylvmgqfsco.10to8.com}.  Appointments may be used for homework help, example problems, polishing of final project ideas, and general help with the course.
\end{enumerate}}
\textit{\textbf{Course Objectives}:}
\begin{itemize}
\item To develop an understanding of the underlying order within theoretical physics
\item To master mathematical physics techniques
\item To develop the ability to apply electromagnetic theory in fields that share similar concepts
\item To appreciate and wield symmetry when constructing mathematical and physical arguments
\item To model complex systems with numerical simulation code
\item To practice developing a shared scientific understanding of scientific results within a small group
\end{itemize}
\textit{\textbf{Course Outline}:}
\begin{outline}[enumerate]
\1 \textbf{Unit 1: Math bootcamp}
\2 Vectors
\2 Vector calculus
\2 Vector fields
\2 \textit{Reading: Chapter 1}
\1 \textbf{Unit 2: Electrostatics}
\2 $\vec{E}$-fields and the divergence and curl of $\vec{E}$-fields
\2 Electric Potential, Work and Energy
\2 Conductors
\2 \textit{Reading: Chapter 2}
\1 \textbf{Unit 3: Potentials}
\2 Laplace's Equation
\2 Separation of variables and Multipole Expansions
\2 \textit{Reading: Chapter 3 (omit 3.2)}
\1 \textbf{Unit 4: Electric Fields in Matter}
\2 Polarized atoms in materials
\2 $\vec{D}$-fields and linear media, part I
\2 \textit{Reading: Chapter 4}
\1 \textbf{Unit 5: Magnetostatics and Magnetic Fields in Matter}
\2 Lorentz-force and the Biot-Savart Law, $\vec{B}$-fields
\2 Divergence and curl of $\vec{B}$-fields
\2 Magnetization and the auxiliary $\vec{H}$-field
\2 Linear media, part II
\2 \textit{Reading: Chapter 5, and 6 (omit 6.4)}
\1 \textbf{Unit 6: Electrodynamics}
\2 EMF, and induction
\2 Maxwell's Equations
\2 \textit{Reading: Chapter 7}
\1 \textbf{Unit 7: Review}
\2 Special topics: numerical simulation of Maxwell's Equations with MEEP
\2 Connection to special relativity
\2 \textbf{Preparation of final projects}
\3 The final project will be given in the form of a presentation to the rest of the class
\3 The final project can be either a theoretical exercise from the book or other source, or a numerical simulation that demonstrates a specific effect
\3 Theoretical final projects should be an explanation of how to do a particularly interesting or difficult calculation
\3 Numerical final projects will involve computational electromagnetism (CEM) simulations
\3 Format: 15 minutes plus questions, submit as PDF file via Moodle
\3 Graded on clarity, correctness, and technical difficulty (equal parts)
\end{outline}
\textbf{Important due Dates}
\begin{outline}
\1 Midterm: Distributed March 18th, 2022.  The midterm is take-home style, and open-book.  The midterm is due no later than March 21st, 2022 by the end of the day.  The midterm should be submitted on Moodle as a PDF file.
\1 Final Project Proposal: A 1-paragraph document proposing the project or calculation to be performed.  The proposal should be submitted as a PDF file on Moodle.  The proposal is due March 28th, 2022 by the end of the day.
\1 Final Project Presentations: We will hold final project presentations on May 11th and May13th, 2022.  Presentations should be 12-15 minutes plus time for questions.
\end{outline}
\end{document}

