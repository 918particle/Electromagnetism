\title{Syllabus for Electromagnetic Theory (PHYS330)}
\author{Dr. Jordan Hanson - Whittier College Dept. of Physics and Astronomy}
\date{\today}
\documentclass[10pt]{article}
\usepackage[a4paper, total={18cm, 27cm}]{geometry}
\usepackage{outlines}
\usepackage{hyperref}
\begin{document}
\maketitle

\begin{abstract}
Maxwell's Equations govern the electromagnetic fields that constitute electrodynamics.  However, an understanding of Maxwell's equations must be built from the fundamental building blocks of vector calculus, bound and free electric charge, electrostatics and electric potential, magnetostatics and magnetic vector potential, and the continuity of free and bound current.  These building blocks will be presented in the traditional order, as shown in \textit{Introduction to Electrodynamics} by D. Griffiths.  Although the course culimates with the presentation of Maxwell's Equations, special topics will be covered along the way.  These include multipole expansions, Fourier transforms and Fourier analysis, and numerical simulation methods.  
\end{abstract}
\noindent
\textit{\textbf{Pre-requisites}:  (PHYS 185/PHYS 250) AND PHYS 275.  This implies Calculus I - III are required.} \\
\textit{\textbf{Course credits, Liberal Arts Categorization}: 3.0 Credits, None} \\
\textit{\textbf{Regular course hours}: MWTRF 13:30-14:45.  \textit{Zoom ID/pass}: 796 092 0745 / 667725.  Most of the textbook content and example problems will be delivered asynchronously. } \\
\textit{\textbf{Instructor contact information}: 
\begin{enumerate}
\item Email: jhanson2@whittier.edu
\item Cell: 562.351.0047
\item Zoom ID / pass: 796 092 0745 / 667725
\item YouTube Channel: \url{www.youtube.com/918particle}
\item Book online appointments: \url{https://fgucmvjkylvmgqfsco.10to8.com}
\end{enumerate}}
\textit{\textbf{Office hours}: Please use the booking service to schedule Zoom meetings for office hours and advising: \url{10to8.com} as above.} \\
\textit{\textbf{Attendance/Absence}: As this is an advanced course, there is no attendance policy. Students will hold themselves accountable, and cannot hope to pass the course if they regularly miss class.} \\ 
\textit{\textbf{Late work policy}: Late work is generally not accepted, but is left to the discretion of the instructor.} \\
\textit{\textbf{Text}: Introduction to Electrodynamics, 4th ed. by D. Griffiths - \url{https://en.wikipedia.org/wiki/Introduction_to_Electrodynamics}}. \\
\textit{\textbf{Grading}: There will be weekly homework sets worth 40 percent of the grade, typically due Fridays.  Homework problems are challenging in this subject, and some collaboration is expected.  There will be weekly quizzes worth 25 percent of the grade.  Quiz problems are designed to test basic understanding, and meant to be worked alone.  There will be no final exam for the course.  Instead, there will be a final project, plus a presentation to the class.  The project will be worth 25 percent of the grade.  Participation in class discussions and examples will be worth 10 percent.} \\
\textit{\textbf{Grade Settings}: $<60\%$ = F, $\geq 60\%, <70\%$ = D, $\geq 70\%, <80\%$ = C, $\geq 80\%, <90\%$ = B, $\geq 90\%, <100\%$ = A.  Pluses and minuses: 0-3\% minus, 3\%-6\% straight, 6\%-10\% plus (e.g. 79\% = C+, 91\% = A-)} \\
\textit{\textbf{Homework Sets}: Typically 5-10 problems per week, assigned and collected on Fridays.  Homeworks must be submitted via Moodle in PDF form.} \\
\textit{\textbf{ADA Statement on Disability Services}: The Americans with Disabilities Act (ADA) is a federal anti-discrimination statute that provides comprehensive civil rights protection for persons with disabilities. Among other things, this legislation requires that all students with disabilities be guaranteed a learning environment that provides for reasonable accommodation of their disabilities. If you believe you have a disability requiring an accommodation, please contact Disability Services: disabilityservices@whittier.edu, tel. 562.907.4825.} \\
\textit{\textbf{Academic Honesty Policy}: \url{http://www.whittier.edu/academics/academichonesty}} \\
\textit{\textbf{Changes due to COVID-19}:
\begin{enumerate}
\item Class will meet daily using the Zoom web-conferencing application, at the normal class time.  Class time will be used to review course status, homework, student questions, and work examples.  The \textit{material content} will be delivered via the reading and video tutorials delivered via YouTube.  This is known as a \textit{flipped} classroom.
\item It is critical that students take advantages of one-on-one meetings with the professor.  Please book an appointment time via 10to8.com: \url{https://fgucmvjkylvmgqfsco.10to8.com}.  Appointments may be used for homework help, example problems, polishing of final project ideas, and general help with the course.
\end{enumerate}}
\textit{\textbf{Course Objectives}:}
\begin{itemize}
\item To master important concepts in theoretical physics.
\item To comprehend the broad applicability of a complete theory of physics
\item To appreciate and wield symmetry when constructing mathematical and physical arguments
\item To model complex systems with numerical simulation code
\item To practice presenting scientific results to a group
\end{itemize}
\textit{\textbf{Course Outline}:}
\begin{outline}[enumerate]
\1 \textbf{Week 1: Math bootcamp}
\2 Vectors
\2 Vector calculus
\2 Vector fields
\2 \textit{Reading: Chapter 1}
\1 \textbf{Week 2: Electrostatics}
\2 $\vec{E}$-fields and the divergence and curl of $\vec{E}$-fields
\2 Electric Potential, Work and Energy
\2 Conductors
\2 \textit{Reading: Chapter 2}
\1 \textbf{Week 3: Potentials}
\2 Laplace's Equation
\2 Separation of variables and Multipole Expansions
\2 \textit{Reading: Chapter 3 (omit 3.2)}
\1 \textbf{Week 4: Electric Fields in Matter}
\2 Polarized atoms in materials
\2 $\vec{D}$-fields and linear media, part I
\2 \textit{Reading: Chapter 4}
\1 \textbf{Week 5: Magnetostatics and Magnetic Fields in Matter}
\2 Lorentz-force and the Biot-Savart Law, $\vec{B}$-fields
\2 Divergence and curl of $\vec{B}$-fields
\2 Magnetization and the auxiliary $\vec{H}$-field
\2 Linear media, part II
\2 \textit{Reading: Chapter 5, and 6 (omit 6.4)}
\1 \textbf{Week 6: Electrodynamics}
\2 EMF, and induction
\2 Maxwell's Equations
\2 \textit{Reading: Chapter 7}
\1 \textbf{Week 7: Review}
\2 Special topics: numerical simulation of Maxwell's Equations with MEEP
\2 Connection to special relativity
\2 \textbf{Preparation of final projects}
\3 The final project will be given in the form of a presentation to the rest of the class
\3 The final project can be either theoretical or numerical
\3 Theoretical final projects will be an explanation of how to do a particularly interesting or difficult calculation
\3 Numerical final projects will be a demonstration of a result of a complex system
\3 Format: 15 minutes plus questions, submit as PDF file via Moodle
\3 Graded on clarity, correctness, and technical difficulty (equal parts)
\end{outline}
\end{document}
