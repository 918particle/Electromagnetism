\title{Warm-up for Electromagnetic Theory (PHYS330)}
\author{Dr. Jordan Hanson - Whittier College Dept. of Physics and Astronomy}
\date{\today}
\documentclass[10pt]{article}
\usepackage[a4paper, total={18cm, 27cm}]{geometry}
\usepackage{hyperref}
\usepackage{amsmath}
\begin{document}
\maketitle

\begin{abstract}
Definition of the Fourier transform, and two interesting results.  These tools may be useful for final projects.
\end{abstract}
\noindent

\section{Definition of the Fourier Transform}

The \textit{Fourier transform} is a way of representing a function of time (or space) as a function of frequency (or wavevector).  Imagine a function of time: $E(t)$ having a partner function in the other space called $\widetilde{E}(\nu)$, where $\nu$ is the frequency.  The Fourier transform turns $E(t)$ into $\widetilde{E}(\nu)$, and the inverse Fourier transform turns $\widetilde{E}(\nu)$ into $E(t)$.  Let $j = \sqrt{-1}$.  Here are the definitions:

\begin{align}
\widetilde{E}(\nu) &= \int_{-\infty}^{\infty} E(t) e^{-2\pi j \nu t} dt \\
E(t) &= \int_{-\infty}^{\infty} \widetilde{E}(\nu) e^{2\pi j \nu t} d\nu
\end{align}

\section{The Fourier transform of the Dirac Delta Function}

Recall the main property of the Dirac delta function, $\delta(t-t_0)$:

\begin{equation}
f(t_0) = \int_{-\infty}^{\infty} f(t) \delta(t - t_0) dt
\end{equation}

In this section, we aim to determine the Fourier transform of a sine wave.  First, \textbf{compute the Fourier transform of the Dirac delta function} by inserting $\delta(t-t_0)$ for $E(t)$ in the definition of the Fourier transform. \\ \\ \textit{[Answer: $e^{-2\pi j \nu t_0}$]} \\ \vspace{1.5cm}

Now, write down the \textit{inverse Fourier transform} of $e^{-2\pi j \nu t_0}$, and simplify the exponential under the integral sign. \\ \vspace{1.5cm}

Finally, in a separate place, write down the \textit{Fourier transform} of $e^{2\pi j \nu_0 t}$, which is equivalent to computing the Fourier transform of a sine wave.

\section{The Fourier transform of a Sine Wave}

Finally, compare your expression for the Fourier transform of $e^{2\pi j \nu_0 t}$ to the inverse Fourier transform of $e^{-2\pi j \nu t_0}$, which was equal to the Dirac delta.  Make the two integrals look as alike as possible.  \textbf{To what is the Fourier transform of} $e^{2\pi j \nu_0 t}$ \textbf{equal}? \\ \vspace{1cm}

Because the solutions to boundary-value problems can be expressed as sums of sines and cosines, you now have the power to express them in \textit{frequency space.}  (There is a minor detail about changing the Fourier transform to relate position and $k$-vector: $f(x)$ goes with $\widetilde{f}(k)$).

\end{document}
